\documentclass[journal,12pt,twocolumn]{IEEEtran}

\usepackage{setspace}
\usepackage{gensymb}
\singlespacing
\usepackage[cmex10]{amsmath}

\usepackage{amsthm}

\usepackage{mathrsfs}
\usepackage{txfonts}
\usepackage{stfloats}
\usepackage{bm}
\usepackage{cite}
\usepackage{cases}
\usepackage{subfig}

\usepackage{longtable}
\usepackage{multirow}

\usepackage{enumitem}
\usepackage{mathtools}
\usepackage{steinmetz}
\usepackage{tikz}
\usepackage{circuitikz}
\usepackage{verbatim}
\usepackage{tfrupee}
\usepackage[breaklinks=true]{hyperref}
\usepackage{graphicx}
\usepackage{tkz-euclide}
\usepackage{float}

\usetikzlibrary{calc,math}
\usepackage{listings}
    \usepackage{color}                                            %%
    \usepackage{array}                                            %%
    \usepackage{longtable}                                        %%
    \usepackage{calc}                                             %%
    \usepackage{multirow}                                         %%
    \usepackage{hhline}                                           %%
    \usepackage{ifthen}                                           %%
    \usepackage{lscape}     
\usepackage{multicol}
\usepackage{chngcntr}

\DeclareMathOperator*{\Res}{Res}

\renewcommand\thesection{\arabic{section}}
\renewcommand\thesubsection{\thesection.\arabic{subsection}}
\renewcommand\thesubsubsection{\thesubsection.\arabic{subsubsection}}

\renewcommand\thesectiondis{\arabic{section}}
\renewcommand\thesubsectiondis{\thesectiondis.\arabic{subsection}}
\renewcommand\thesubsubsectiondis{\thesubsectiondis.\arabic{subsubsection}}


\hyphenation{op-tical net-works semi-conduc-tor}
\def\inputGnumericTable{}                                 %%

\lstset{
%language=C,
frame=single, 
breaklines=true,
columns=fullflexible
}
\begin{document}

\newcommand{\BEQA}{\begin{eqnarray}}
\newcommand{\EEQA}{\end{eqnarray}}
\newcommand{\define}{\stackrel{\triangle}{=}}
\bibliographystyle{IEEEtran}
\raggedbottom
\setlength{\parindent}{0pt}
\providecommand{\mbf}{\mathbf}
\providecommand{\pr}[1]{\ensuremath{\Pr\left(#1\right)}}
\providecommand{\qfunc}[1]{\ensuremath{Q\left(#1\right)}}
\providecommand{\sbrak}[1]{\ensuremath{{}\left[#1\right]}}
\providecommand{\lsbrak}[1]{\ensuremath{{}\left[#1\right.}}
\providecommand{\rsbrak}[1]{\ensuremath{{}\left.#1\right]}}
\providecommand{\brak}[1]{\ensuremath{\left(#1\right)}}
\providecommand{\lbrak}[1]{\ensuremath{\left(#1\right.}}
\providecommand{\rbrak}[1]{\ensuremath{\left.#1\right)}}
\providecommand{\cbrak}[1]{\ensuremath{\left\{#1\right\}}}
\providecommand{\lcbrak}[1]{\ensuremath{\left\{#1\right.}}
\providecommand{\rcbrak}[1]{\ensuremath{\left.#1\right\}}}
\theoremstyle{remark}
\newtheorem{rem}{Remark}
\newcommand{\sgn}{\mathop{\mathrm{sgn}}}
\providecommand{\abs}[1]{\vert#1\vert}
\providecommand{\res}[1]{\Res\displaylimits_{#1}} 
\providecommand{\norm}[1]{\lVert#1\rVert}
%\providecommand{\norm}[1]{\lVert#1\rVert}
\providecommand{\mtx}[1]{\mathbf{#1}}
\providecommand{\mean}[1]{E[ #1 ]}
\providecommand{\fourier}{\overset{\mathcal{F}}{ \rightleftharpoons}}
%\providecommand{\hilbert}{\overset{\mathcal{H}}{ \rightleftharpoons}}
\providecommand{\system}{\overset{\mathcal{H}}{ \longleftrightarrow}}
	%\newcommand{\solution}[2]{\textbf{Solution:}{#1}}
\newcommand{\solution}{\noindent \textbf{Solution: }}
\newcommand{\cosec}{\,\text{cosec}\,}
\providecommand{\dec}[2]{\ensuremath{\overset{#1}{\underset{#2}{\gtrless}}}}
\newcommand{\myvec}[1]{\ensuremath{\begin{pmatrix}#1\end{pmatrix}}}
\newcommand{\mydet}[1]{\ensuremath{\begin{vmatrix}#1\end{vmatrix}}}
\numberwithin{equation}{subsection}
\makeatletter
\@addtoreset{figure}{problem}
\makeatother
\let\StandardTheFigure\thefigure
\let\vec\mathbf
\renewcommand{\thefigure}{\theproblem}
\def\putbox#1#2#3{\makebox[0in][l]{\makebox[#1][l]{}\raisebox{\baselineskip}[0in][0in]{\raisebox{#2}[0in][0in]{#3}}}}
     \def\rightbox#1{\makebox[0in][r]{#1}}
     \def\centbox#1{\makebox[0in]{#1}}
     \def\topbox#1{\raisebox{-\baselineskip}[0in][0in]{#1}}
     \def\midbox#1{\raisebox{-0.5\baselineskip}[0in][0in]{#1}}

\vspace{3cm}

\title{AI1103-Assignment 2}
\author{Name: Ramanathan Annamalai\\\normalsize{Roll Number: BM20BTECH11011\\\includegraphics[scale=0.34]{IITHLogo.png}}}
\maketitle
\newpage
\bigskip
\renewcommand{\thefigure}{\theenumi}
\renewcommand{\thetable}{\theenumi}
Download all python codes from 
\begin{lstlisting}
https://github.com/Ramanathan-Annamalai/AI1103-Probability_and_Random_Variables/tree/main/Assignment%202/Codes
\end{lstlisting}
%
and latex-tikz codes from 
%
\begin{lstlisting}
https://github.com/Ramanathan-Annamalai/AI1103-Probability_and_Random_Variables/blob/main/Assignment%202/Assignment_2.tex
\end{lstlisting}
\section*{\large{Question}}
Suppose we uniformly and randomly select a permutation from the 20\(!\) permutations of 1,2,3,\ldots,20. What is the probability that 2 appears at an earlier position than any other even number in the selected permutation.
\begin{enumerate}[label=(\Alph*)]
    \item \(\displaystyle\frac{1}{2}\)\newline
    \item \(\displaystyle\frac{1}{10}\)\newline
    \item \(\displaystyle\frac{9!}{20!}\)\newline
    \item \normalsize{None of the above.}\newline
\end{enumerate}

\section*{\large{Solution}}
Total number of permutations = \(T\)
\begin{align}
    T = 20!
\end{align}

No. of ways of choosing 10 places for the odd numbers 
\begin{align}
    = {}^{20}C_{10}
\end{align}

No. of ways of arranging the odd numbers 
\begin{align}
    = 10!
\end{align}

Since 2 should occur before other even numbers, the first blank place in the permutation has to be 2.\newline

No. of ways of arranging the even numbers other than 2 in the remaining 9 places 
\begin{align}
    = 9!
\end{align}

Number of favourable permutations 
\begin{align}
    = {}^{20}C_{10}\times10!\times9!\\\nonumber
\end{align}

\begin{tikzpicture}[squarednode_odd/.style={rectangle, draw=red!60, minimum width=2.6mm, node distance = 0.1 pt, inner sep = 2.9 },
squarednode_even/.style={rectangle, draw=blue!60, minimum width=1.5mm, node distance = 0.1 pt, inner sep = 2.9 },
squarednode_2/.style={rectangle, draw=green!60,very thick, minimum width=2.6mm, node distance = 0.1 pt, inner sep = 2.9}]
    \node[rectangle, draw=red!0, minimum width=2.6mm, node distance = 0.1 pt, inner sep = 1 ]     (0)     {};
    \node[squarednode_odd]       (1)         [right=of 0]                 {o};
    \node[squarednode_odd]       (2)          [right=of 1]              {o};
    \node[squarednode_odd]       (3)          [right=of 2]              {o};
    \node[squarednode_2]         (4)         [right=of 3]              {2};
    \node[squarednode_even]      (5)         [right=of 4]              {e};
    \node[squarednode_odd]       (6)          [right=of 5]              {o};
    \node[squarednode_even]      (7)         [right=of 6]              {e};
    \node[squarednode_even]      (8)         [right=of 7]              {e};
    \node[squarednode_odd]       (9)          [right=of 8]              {o};
    \node[squarednode_odd]       (10)          [right=of 9]              {o};
    \node[squarednode_even]      (11)         [right=of 10]              {e};
    \node[squarednode_even]      (12)         [right=of 11]              {e};
    \node[squarednode_odd]       (13)          [right=of 12]              {o};
    \node[squarednode_even]      (14)         [right=of 13]              {e};
    \node[squarednode_odd]       (15)          [right=of 14]              {o};
    \node[squarednode_odd]       (16)          [right=of 15]              {o};
    \node[squarednode_even]      (17)         [right=of 16]              {e};
    \node[squarednode_even]      (18)         [right=of 17]              {e};
    \node[squarednode_even]      (19)         [right=of 18]              {e};
    \node[squarednode_odd]       (20)          [right=of 19]              {o};
    
    \node[rectangle ,draw=red!60, minimum width=2.6mm, inner sep = 3 , outer sep = 1] (o) [below=of 3] {o};
    \node[rectangle , node distance = 20pt, inner sep = 3 , outer sep = 1] (legendodd) [right=of o] {choose positions for odd numbers};
    \node[rectangle ,draw=green!60,very thick, minimum width=2.6mm ,node distance = 10pt, inner sep = 3 , outer sep = 1] (two) [below=of o] {2};
    \node[rectangle , node distance = 20pt, inner sep = 3 , outer sep = 1] (legendtwo) [right=of two] {place 2 in the first empty space};
    \node[rectangle ,draw=blue!60, minimum width=2.6mm , node distance = 10pt, inner sep = 3 , outer sep = 1] (e) [below=of two] {e};
    \node[rectangle , node distance = 20pt, inner sep = 3 , outer sep = 1] (legendeven) [right=of e] {place remaining even numbers};
    \node[rectangle , draw red!0] (empty) [below=of e] {};
    
    \draw[->] (o.east) -- (legendodd.west);
    \draw[->] (e.east) -- (legendeven.west);
    \draw[->] (two.east) -- (legendtwo.west);
\end{tikzpicture}


Now, Probability of 2 appears at an earlier position than any other even number in the selected permutation = \(p\)\newline
\begin{align}
    p &= \frac{\text{No. of favourable permutations}}{\text{Total no. of permutations}}\nonumber\\
    &= \frac{{}^{20}C_{10}\times10!\times9!}{20!}\\
    &= \frac{1}{10}\\\nonumber
\end{align}

\begin{flushright}
    Answer: \textbf{Option (B)}.
\end{flushright}
\end{document}
